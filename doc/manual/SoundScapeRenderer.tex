\documentclass[a4paper]{scrartcl}

\usepackage[sumlimits,intlimits]{amsmath}
%\usepackage{array}
\usepackage{wasysym}
\usepackage{graphicx}
\graphicspath{{images/}}
\usepackage{subfigure}
\usepackage{psfrag}
\usepackage{nicefrac}
\usepackage{comment}
\usepackage[colorlinks,urlcolor=black,linkcolor=black,citecolor=black]{hyperref}

\newcommand{\contactadress}{\emph{SoundScapeRenderer@telekom.de}}

\begin{document}

\sloppy

\title{\Huge Introduction to the\\SoundScape Renderer (SSR)}
\author{Jens Ahrens, Matthias Geier and Sascha Spors\\[2ex]
\contactadress}

\date{\today}

\maketitle

\centerline{\includegraphics[width=.4\linewidth]{ssr_logo.mps}}

\begin{abstract}
\noindent\textsc{\Large
The SoundScape Renderer (SSR) comes with \mbox{ABSOLUTELY} NO WARRANTY.
The SSR is free software and released under the GNU
General Public License, either version 3 of the License, or (at your option)
any later version. For details, see the enclosed file COPYING.
}
\vspace{\baselineskip}

\begin{tabular}{ll}
Copyright \textcopyright\ 2006--2012 & Quality \& Usability Lab\\
 & Deutsche Telekom Laboratories\\
 & Technische Universit\"at Berlin\\
 & Ernst-Reuter-Platz 7, 10587 Berlin, Germany\\
% & \url{http://qu.tu-berlin.de}
\end{tabular}
\end{abstract}

\thispagestyle{empty} % no page number on first page

% include your stuff here

\newpage
\tableofcontents

\include{general}
\include{operation}
\include{renderers}
\include{gui}
\include{network}
%\include{todo}

\bibliographystyle{unsrt}
\bibliography{references}

\end{document}
