\section{Compiling and running the SSR}

\subsection{Mac OSX App-Bundle}

The following sections are relevant if you want to build the SSR from its source
code. This is the default procedure for GNU/Linux systems. If you want to use
the SSR on a Mac, you can also use the pre-compiled app bundle.
For further information visit the SSR development website%
\footnote{\url{http://dev.qu.tu-berlin.de/projects/ssr/wiki/SSR_for_MacOSX}}.

\subsection{Getting the Source}

If you didn't receive this manual with the source code of the SSR, you can
download it from the SSR website~\cite{ssr}. After downloading, you can unpack the tarball with
the command \verb+tar xvzf ssr-x.x.x.tar.gz+ in a shell. This will extract the
source code to a directory of the form \verb+ssr-x.x.x+ where ``x'' stands for
the version numbers.  \verb+cd+ to this directory and proceed with section
\ref{sec:configuring} to configure the SSR.

\subsection{Configuring}
\label{sec:configuring}

To build the SSR from source you have to configure first. Open a shell and
\verb+cd+ to the directory containing the source code
%\footnote{Note that the
%top directory of the SSR-package distributed e.g. in a tar-ball is intended.}
of the package and type:

\begin{verbatim}
./configure
\end{verbatim}

This script will check your system for dependencies and prepare the
\verb+Makefile+ required for compilation. Section \ref{sec:dependencies} lists
the dependencies that must be installed on your system. The \texttt{configure} script
will signal if dependencies are missing. At successful termination of the 
\texttt{configure} script a summary will show up.

Section \ref{sec:hints_conf} is intended to help you troubleshooting.

\subsubsection{Dependencies}
\label{sec:dependencies}
At least the following software (libraries and headers) including their
development packages (\emph{dev} or \emph{devel}),
where available, are required for a full installation of the SSR:

\begin{itemize}
\item[-] JACK Audio Connection Kit \cite{jack}
\item[-] FFTW3 compiled for single precision (\texttt{fftw3f}) version 3.0 or higher
  \cite{fftw3}
\item[-] libsndfile \cite{sndfile}
\item[-] Ecasound \cite{ecasound}
\item[-] Trolltech's Qt 4.2.2 or higher with OpenGL (QtCore, QtGui and QtOpenGL)
  \cite{qt4}
%\item[-] GLUT \cite{glut} or freeglut \cite{freeglut}
\item[-] libxml2 \cite{libxml2}
\item[-] Boost.Asio \cite{boost}, included since Boost version 1.35.
\end{itemize}

We provide a simple integration of two tracking systems.
Please read section
%\ref{sec:configuring} for an actual configuration or read section
\ref{sec:head_tracking} for further informations about head tracking.

\subsubsection{Hints on Configuration}
\label{sec:hints_conf}

If you encounter problems configuring the SSR these hints could help:
\begin{itemize}
\item Ensure that you really installed all libraries (\verb+lib+) with devel-package (\verb+devel+ or \verb+dev+, where available) mentioned in section \ref{sec:dependencies}.
\item It may be necessary to run \verb+ldconfig+ after installing new libraries.
\item Ensure that \verb+/etc/ld.so.conf+ or \verb+LD_LIBRARY_PATH+ are set
  properly, and run \verb+ldconfig+ after changes.
\item If a header is not installed in the standard paths of your system you
  can pass its location to the configure script using
  \verb+./configure CPPFLAGS=-Iyourpath+.
\end{itemize}

Note that with \verb+./configure --help+ all configure-options are displayed,
e.g. in section ``Optional Features'' you will find how to
%enable optimization by compiler, how to
disable compilation of the head trackers
%or how to start the SSR with a different renderer.
and many other things.
Setting the influential environment variables with
\verb+./configure VARNAME=value+ can be useful for debugging dependencies.

\subsection{Compiling and Installing}
\label{sec:comp_inst}

If the configure script terminates with success, it creates a file named 
\texttt{Makefile}. You can build the SSR by typing

\begin{quote}
\texttt{make}\\
\texttt{make install}
\end{quote}
%
This will compile the SSR and install it to your system. We recommend the usage of GCC 4 or higher. Note that with \verb+./configure --prefix+ you can specify
where to install the SSR on your system. The default prefix is
\texttt{/usr/local}.
%Section \ref{sec:configuring} contains more information about configuring.

\subsection{Uninstalling}

If the SSR didn't meet your expectations, we are very sorry, and of course you
can easily remove it from your system with
\begin{quote}
\texttt{make uninstall}
\end{quote}

\subsection{Running the SSR}
\label{sec:running_ssr}

Before you start the SSR, start JACK \cite{jack}, e.g.~by typing\\
\verb+jackd -d alsa -r 44100+ in a shell or using the graphical user
interface ``qjackctl'' \cite{qjackctl}.
Now, the easiest way to get a signal out of the SSR is
by passing a sound-file directly:

\begin{quote}
\begin{verbatim}
ssr YOUR_AUDIO_FILE
\end{verbatim}
\end{quote}

By default, the SSR starts with the binaural renderer; please use
headphones for listening with this renderer.
Type \verb+ssr --help+ to get an overview of the command
line options and various renderers:

{\small
\begin{verbatim}
USAGE: ssr [OPTIONS] <scene-file>

The SoundScape Renderer (SSR) is a tool for real-time spatial audio reproduction
providing a variety of rendering algorithms.

OPTIONS:

Choose a rendering algorithm:
    --binaural         Binaural (using HRIRs)
    --brs              Binaural Room Synthesis (using BRIRs)
    --bpb              Binaural Playback Renderer
    --wfs              Wave Field Synthesis
    --aap              Ambisonics Amplitude Panning
    --vbap             Stereophonic (Vector Base Amplitude Panning)
    --generic          Generic Renderer

Renderer-specific options:
    --hrirs=FILE       Load the HRIRs for binaural renderer from FILE
    --hrir-size=VALUE  Maximum IR length (binaural and BRS renderer)
    --prefilter=FILE   Load WFS prefilter from FILE
-o, --ambisonics-order=VALUE Ambisonics order to use (default: maximum)
    --in-phase-rendering     Use in-phase rendering for Ambisonics

JACK options:
    --input-prefix=PREFIX    Input  port prefix (default: "system:capture_")
    --output-prefix=PREFIX   Output port prefix (default: "system:playback_")
-f, --freewheel        Use JACK in freewheeling mode

General options:
-c, --config=FILE      Read configuration from FILE
-s, --setup=FILE       Load reproduction setup from FILE
-r, --record=FILE      Record the audio output of the renderer to FILE
    --loop             Loop all audio files
    --master-volume-correction=VALUE
                       Correction of the master volume in dB (default: 0 dB)
-i, --ip-server[=PORT] Start IP server (default on)
                       A port can be specified: --ip-server=5555
-I, --no-ip-server     Don't start IP server
-g, --gui              Start GUI (default)
-G, --no-gui           Don't start GUI
-t, --tracker=TYPE     Start tracker, possible value(s): polhemus intersense razor
    --tracker-port=PORT
                       A serial port can be specified, e.g. /dev/ttyS1
-T, --no-tracker       Don't start tracker

-h, --help             Show this very help information. You just typed that!
-v, --verbose          Increase verbosity level (up to -vvv)
-V, --version          Show version information and exit
\end{verbatim}
}

Choose the appropriate arguments and make sure that your amplifiers
are not turned too loud\dots

To stop the SSR use either the options provided by the GUI (section
\ref{sec:gui}) or type \texttt{Crtl+c} in the shell in which you started the
SSR.

\paragraph{Keyboard actions in non-GUI mode}

If you start SSR without GUI (option \verb+--no-gui+), it starts automatically
replaying the scene you have loaded. You can have some interaction via the
shell. Currently implemented actions are (all followed by \texttt{Return}):

\begin{itemize}
\item[] \texttt{c}: calibrate tracker (if available)
\item[] \texttt{p}: start playback
\item[] \texttt{q}: quit application
\item[] \texttt{r}: ``rewind''; go back to the beginning of the current scene
\item[] \texttt{s}: stop (pause) playback
\end{itemize}
%
Note that in non-GUI mode, audio processing is always taking place. Live inputs
are processed even if you pause playback.

\paragraph{Recording the SSR output}

You can record the audio output of the SSR using the \texttt{--record=FILE} command
line option. All output signals (i.e.\ the loudspeaker signals) will be recorded
to a multichannel wav-file named \texttt{FILE}. The order of channels
corresponds to the order of loudspeakers specifed in the reproduction setup
(see sections \ref{sec:reproduction_setups} and \ref{sec:asdf}). The recording
can then be used to analyze the SSR output or to replay it without the SSR
using a software player like \texttt{ecaplay}~\cite{ecasound}.

\subsection{Configuration File}
\label{sec:ssr_configuration_file}

The general configuration of the SSR (that means which renderer to use, if GUI
is enabled etc.) can be specified in a configuration file
(e.g. \texttt{ssr.conf}). By specifying
your wishes in such a file, you avoid having to give explicit command
line options every time you start the SSR. We have added the example
\texttt{data/ssr.conf.example} which mentions all possible parameters.
Take a look inside, it is rather self-explanatory.
There are three possibilities to specify a configuration file:
\begin{itemize}
\item put it in \texttt{/etc/ssr.conf}
\item put it in your home directory in \texttt{\$HOME/.ssr/ssr.conf}
\item specify it on the command line with \texttt{ssr -c my\_config.file}
\end{itemize}

We explicitly mention one parameter here which might be of immediate interest
for you: \texttt{MASTER\_VOLUME\_CORRECTION}. This a correction in dB~(!) which is
applied -- as you might guess -- to the master volume. The motivation is to have
means to adopt the general perceived loudness of the reproduction of a given
system. Factors like the distance of the loudspeakers to the listener or the
typical distance of virtual sound sources influence the resulting loudness
which can be adjusted to the desired level by means of the
\texttt{MASTER\_VOLUME\_CORRECTION}. Of course, there's also a command line
alternative (\texttt{--master-volume-correction}).

\subsection{Head Tracking}
% TODO: update this section.
\label{sec:head_tracking}

We provide integration of the \emph{InterSense InertiaCube3} tracking sensor
\cite{intersense} and the  \emph{Polhemus Fastrak}~\cite{fastrack}. They are used to
update the orientation of the reference (in binaural reproduction this is the
listener) in real-time. Please read sections \ref{sec:prep_isense} and
\ref{sec:prep_pol} if you want to compile the SSR with the support for these
trackers.

Note that on startup, the SSR tries to find the tracker. If it fails, it
continues without it. If you use a tracker, make sure that you have the
appropriate rights to read from the respective port.

You can calibrate the tracker while the SSR is running by pressing
\texttt{Return}.
The instantaneous
orientation will then be interpreted as straight forward ($\alpha = 90^\circ$).

\subsubsection{Preparing InterSense InertiaCube3}
\label{sec:prep_isense}

If you want to compile the SSR with support for the \emph{InterSense
InertiaCube3}
tracking sensor~\cite{intersense}, please download the \emph{InterSense Software
Development Kit} (SDK) from the InterSense website~\cite{intersense}.
Unpack the archive and place the files

\begin{itemize}
\item \verb+isense.h+ and \verb+types.h+ to \verb+/usr/local/include+, and
\item \verb+libisense.so+
  %(either from folder \texttt{x86} or \texttt{x86\_64})
  (the version appropriate for your processor type)
  to \verb+usr/local/lib+.
\end{itemize}
%
The SSR \texttt{configuration} script will automatically detect the presence of
the files described above and if they are found, enable the compilation for the 
support of this tracker. To disable this tracker, use 
\verb+./configure --disable-intersense+ and recompile.

If you encounter an error-message similar to \texttt{libisense.so: cannot open 
shared object file: No such file or directory}, but the file is placed 
correctly, run \verb|ldconfig|. 

\subsubsection{Preparing Polhemus Fastrack}
\label{sec:prep_pol}
For incorporation of the \emph{Polhemus Fastrack}~\cite{fastrack} with serial
connection, no additional libraries are required.
%The files
%\verb+termios.h,unistd.h, fcntl.h and poll.h+ should be already installed on
%your system. If they are found, the configure-script will enable the
%compilation for the support of this tracker.
If you want to disable this
tracker, use \verb+ ./configure --disable-polhemus+ and recompile.

\subsection{Using the SSR with DAWs}

As stated before, the SSR is currently not able to dynamically replay audio
files (refer to section~\ref{sec:asdf}). If your audio scenes are complex, you
might want to consider using the SSR together with a digital audio work station
(DAW). To do so, you simply have to create as many sources in the SSR as you
have audio tracks in your respective DAW project and assign live inputs to the
sources. Amongst the ASDF examples we provide at~\cite{ssr} you find an example
scene description which does exactly this.

DAWs like Ardour~\cite{ardour} support JACK and their use is therefore straightforward. DAWs which do not run on Linux or do not support JACK can be connected
via the input of the sound card.

In the future we will provide a VST plug-in which will allow you to dynamically
operate all virtual source's properties (like e.g.~a source's position or level
etc.). You will then be able to have the full SSR functionality controlled from your
DAW.
